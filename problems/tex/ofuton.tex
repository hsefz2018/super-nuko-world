\documentclass[UTF8, 11pt, a4paper]{article}
\usepackage[cm]{sfmath}
\usepackage{tabularx}
\def\arraystretch{1.3}
\usepackage[a4paper, top=3.18cm,bottom=3.81cm,left=2.54cm,right=2.54cm]{geometry}
\usepackage{indentfirst}
\setlength{\parskip}{6pt}
\XeTeXlinebreaklocale "zh"
\usepackage{graphicx}
\usepackage[normalem]{ulem}

\usepackage{fontspec}
\setmainfont{思源黑体}
\SetSymbolFont{largesymbols}{normal}{OMX}{iwona}{m}{n}
\setmonofont{Source Code Pro}

\begin{document}
\section*{蒲团 / Ofuton}

\subsection*{描述}
Nuko 家不远处有一块很大很大的平地,上面有 $N$ 个猫窝,每个猫窝可以看作平面上的一个点。

冬天马上就要到了,Nuko 有一些多余的布料和棉花,打算利用它们制作一种%
叫做“蒲团”的东西,把猫窝盖得严严实实的,以帮助猫咪抵御寒冷。%
但是由于蒲团太大了,Nuko 只能把它们做成长方形,而且两个蒲团不可以重叠,即使只是%
有公共边也不行。蒲团还不方便旋\textbf{倦},%
为了方便起见,Nuko 在这块平地上建立了一个直角坐标系,%
放置蒲团的时候四条边都要和横纵坐标轴平行。这样一来,一块蒲团就可以盖住%
在这个长方形内(\uwave{包括边界})的所有猫窝了。蒲团还可以做成长条形,这个时候%
它有些边的长度是 0,所需材料的面积也是 0,但是仍然可以覆盖它所在直线上的猫窝。

为了尽快实现自己的愿望,Nuko 计划用一大块面积尽可能小的蒲团把所有的猫窝都盖住。%
一旁的 Mafu 很快发现这样太浪费了,又会增加制作的时间而导致得不偿失,于是便有了一个伟大想法%
——用两块蒲团而不是一块!为了说服 Nuko,Mafu 想要知道,如果用两块蒲团%
来覆盖所有的猫窝,最多可以节省多少材料的面积。

\subsection*{输入 \makebox[0.5em]{} \small{ofuton.in}}
\begin{itemize}
    \item 第 1 行:一个正整数 $N$。
    \item 接下来 $N$ 行:每行包含两个正整数 $X_i$ 和 $Y_i$,表示第 $i$ 个猫窝的坐标。
\end{itemize}

\subsection*{输出 \makebox[0.5em]{} \small{ofuton.out}}
\begin{itemize}
    \item 第 1 行:一个正整数,表示用两块蒲团代替原来的一块所节省的最大面积。
\end{itemize}

\subsection*{样例}
\begin{table}[h]\centering
\begin{tabularx}{0.8 \textwidth}{|X|X|}
\hline
\texttt{\textbf{ofuton.in}} & \texttt{\textbf{ofuton.out}} \\ \hline
{\ttfamily
6\newline
4 2\newline
8 10\newline
1 1\newline
9 12\newline
14 7\newline
2 3
} & {\ttfamily
107
}
\\ \hline
\end{tabularx}\end{table}
\subsubsection*{说明}
在 Nuko 的原计划中,蒲团的面积是 143;

在 Mafu 的计划中,可以用一块面积为 6 的蒲团盖住第 1、3、6 个猫窝,%
再用一块面积为 30 的蒲团盖住第 2、4、5 个猫窝。可以证明这是最优方案,%
节省的材料面积为 143 - 30 - 6 = 107。

\subsection*{数据规模}
\begin{table}[h]\centering
\begin{tabularx}{0.5 \textwidth}{X|X|X}
子任务 & pts & $N$            \\ \hline\hline
1      & 6   & $= 2$          \\ \hline
2      & 19  & $\leq 20$      \\ \hline
3      & 35  & $\leq 3\,000$  \\ \hline
4      & 40  & $\leq 50\,000$ \\
\end{tabularx}\end{table}
另外,对于所有的数据,满足 $N \geq 2$,$1 \leq X_i, Y_i \leq 10^9$。

\subsection*{限制}
\begin{itemize}
\item 时间:1.0 秒
\item 内存:256 MiB
\end{itemize}

\begin{figure}[h]\centering
\includegraphics[width=0.6 \textwidth]{ofuton.png}
\end{figure}

\end{document}

