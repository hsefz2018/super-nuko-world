\documentclass[UTF8, 11pt, a4paper]{article}
\usepackage[cm]{sfmath}
\usepackage{tabularx}
\def\arraystretch{1.3}
\usepackage[a4paper, top=3.18cm,bottom=3.81cm,left=2.54cm,right=2.54cm]{geometry}
\usepackage{indentfirst}
\setlength{\parskip}{6pt}
\XeTeXlinebreaklocale "zh"
\usepackage{graphicx}
\usepackage[normalem]{ulem}

\usepackage{fontspec}
\setmainfont{思源黑体}
\SetSymbolFont{largesymbols}{normal}{OMX}{iwona}{m}{n}
\setmonofont{Source Code Pro}

\begin{document}
\section*{狗尾草 / Nekojyarashi}

\subsection*{描述}
Nuko 一直用菜园里冒出来的狗尾草去逗旁边草坪上的猫咪,致使它们产生了抗性,%
狗尾草很快就要失去作用了。

但是最近 Nuko 有了激动人心的新发现:一个存在于狗尾草上被称为 “Nyan 环” 的东西,%
这个发现简直可以造个大新闻!%
简而言之,一棵狗尾草可以由一些节点和它们之间的无向边来描述,一个节点可以%
连出任意多条边,但是不会存在重边(重复的边)和自环(连向自己的边)。%
一个 Nyan 环是一个包括奇数个节点(至少 3 个)的简单环(每个节点至多只经过一次的环)。%
Nuko 经过研究发现带有 Nyan 环的狗尾草可以大大增加猫从猫窝里出来的几率。

Mafu 刚从 Nuko 的园子里坑出来一大把狗尾草,也打算利用 Nyan 环把猫咪引出来玩。%
Mafu 经过另外的研究,学会了把两个狗尾草节点连接起来的办法。%
急切地想要实现这个愿望的 Mafu 打算在这一团狗尾草之间制造出 Nyan 环。%
Mafu 很想知道下面两个问题的答案:
\begin{enumerate}
    \item 在这团狗尾草里制造一个 Nyan 环至少需要添加多少条边;
    \item 有多少种这样的最优方案。
\end{enumerate}

\subsection*{输入 \makebox[0.5em]{} \small{nekojyarashi.in}}
\begin{itemize}
    \item 第 1 行:两个正整数 $N$、$M$,分别表示这团狗尾草的节点数和边数。
    \item 接下来 $M$ 行:每行包含两个正整数 $u_i$、$v_i$,%
        表示节点 $u_i$ 和 $v_i$ 之间有一条无向边连接。%
        节点从 1 编号至 $N$。整张图\uwave{不}保证联通。
\end{itemize}

\subsection*{输出 \makebox[0.5em]{} \small{nekojyarashi.out}}
\begin{itemize}
    \item 第 1 行:一个整数,表示为了形成 Nyan 环需要添加的最少边数 $T$。
    \item 第 2 行:一个整数,表示添加 $T$ 条边形成至少一个 Nyan 环的不同方案数 $W$。
        两种方案不同当且仅当它们添加的边的集合不同。
\end{itemize}

\subsection*{样例}
\begin{table}[h]\centering
\begin{tabularx}{0.8 \textwidth}{|X|X|}
\hline
\texttt{\textbf{nekojyarashi1.in}} & \texttt{\textbf{nekojyarashi1.out}} \\ \hline
{\ttfamily
4 4\newline
1 2\newline
1 3\newline
4 2\newline
4 3
} & {\ttfamily
1\newline
2
}
\\ \hline
\end{tabularx}\end{table}
\subsubsection*{样例一 说明}
在 (2, 3) 或者 (1, 4) 之间添加一条边都可以形成长度为 3 的环(而且还不止一个!OvO)
\newline\newline

\begin{table}[h]\centering
\begin{tabularx}{0.8 \textwidth}{|X|X|}
\hline
\texttt{\textbf{nekojyarashi2.in}} & \texttt{\textbf{nekojyarashi2.out}} \\ \hline
{\ttfamily
3 3\newline
1 2\newline
2 3\newline
3 1
} & {\ttfamily
0\newline
1
}
\\ \hline
\end{tabularx}\end{table}
\subsubsection*{样例二 说明}
不用添加任何边,1--2--3 本身就是一个包含 3 个节点的 Nyan 环。
\newline\newline

\begin{table}[h]\centering
\begin{tabularx}{0.8 \textwidth}{|X|X|}
\hline
\texttt{\textbf{nekojyarashi3.in}} & \texttt{\textbf{nekojyarashi3.out}} \\ \hline
{\ttfamily
3 0
} & {\ttfamily
3\newline
1
}
\\ \hline
\end{tabularx}\end{table}

\subsection*{数据规模}
\begin{table}[h]\centering
\begin{tabularx}{0.8 \textwidth}{X|X|X|X}
子任务 & pts & $N$             & 附加条件 \\ \hline\hline
1      & 10  & $= 3$           & - \\ \hline
2      & 5   & $\leq 100\,000$ & $M \leq 1$ \\ \hline
3      & 20  & $\leq 100$      & - \\ \hline
4      & 15  & $\leq 1\,000$   & - \\ \hline
5      & 17  & $\leq 100\,000$ & 每个点连出的边数不超过 2 \\ \hline
6      & 33  & $\leq 100\,000$ & - \\
\end{tabularx}\end{table}
另外,对于所有的数据,保证 $N \geq 3$,$M \leq \min\{\frac{n(n-1)}{2}, 100\,000\}$,%
不存在重边和自环,\uwave{不保证}图联通!!

\subsubsection*{特殊的评分方式}
本题采用 Special Judge 对程序的输出进行评分。具体来说,对于每个测试点:
\begin{itemize}
    \item 如果程序输出格式错误(包括但不限于输出非数字、%
        输出浮点数如 “3.00”、输出 “QAQ”),评测机会表示很不高兴,将该测试点判作 0 分;
    \item 其他情况下,如果程序输出的第一问答案 $T$ 不正确,该测试点获得 0\% 的分数;
    \item 其他情况下,如果程序输出的第二问答案 $W$ 不正确,该测试点获得 30\% 的分数;
    \item 其他情况下,该测试点获得 100\% 的分数。
\end{itemize}
对于每个子任务,获得的分数是程序在该子任务下所有测试点得分%
的最小值与该子任务总分的乘积。本题获得的总分是所有子任务得分之和%
四舍五入取整后的值。

\subsection*{限制}
\begin{itemize}
\item 时间:1.0 秒
\item 内存:256 MiB
\end{itemize}

\begin{figure}[h]\centering
\includegraphics[width=0.6 \textwidth]{nekojyarashi.png}
\end{figure}

\end{document}

